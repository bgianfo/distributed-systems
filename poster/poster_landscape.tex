\documentclass[landscape,a0paper,final,showframe]{baposter}

\usepackage{times}
\usepackage{calc}
\usepackage{graphicx}
\usepackage{amsmath}
\usepackage{amssymb}
\usepackage{relsize}
\usepackage{multirow}
\usepackage{bm}
\usepackage{xcolor}
\usepackage{graphicx}
\usepackage{multicol}

\usepackage{pgfbaselayers}
\pgfdeclarelayer{background}
\pgfdeclarelayer{foreground}
\pgfsetlayers{background,main,foreground}

\usepackage{helvet}
%\usepackage{bookman}
\usepackage{palatino}

\newcommand{\captionfont}{\footnotesize}

\selectcolormodel{cmyk}

\graphicspath{{images/}}

%%%%%%%%%%%%%%%%%%%%%%%%%%%%%%%%%%%%%%%%%%%%%%%%%%%%%%%%%%%%%%%%%%%%%%%%%%%%%%%%
%%%% Some math symbols used in the text
%%%%%%%%%%%%%%%%%%%%%%%%%%%%%%%%%%%%%%%%%%%%%%%%%%%%%%%%%%%%%%%%%%%%%%%%%%%%%%%%
% Format 
\newcommand{\Matrix}[1]{\begin{bmatrix} #1 \end{bmatrix}}
\newcommand{\Vector}[1]{\Matrix{#1}}
\newcommand*{\SET}[1]  {\ensuremath{\mathcal{#1}}}
\newcommand*{\MAT}[1]  {\ensuremath{\mathbf{#1}}}
\newcommand*{\VEC}[1]  {\ensuremath{\bm{#1}}}
\newcommand*{\CONST}[1]{\ensuremath{\mathit{#1}}}
\newcommand*{\norm}[1]{\mathopen\| #1 \mathclose\|}% use instead of $\|x\|$
\newcommand*{\abs}[1]{\mathopen| #1 \mathclose|}% use instead of $\|x\|$
\newcommand*{\absLR}[1]{\left| #1 \right|}% use instead of $\|x\|$

\def\norm#1{\mathopen\| #1 \mathclose\|}% use instead of $\|x\|$
\newcommand{\normLR}[1]{\left\| #1 \right\|}% use instead of $\|x\|$

%%%%%%%%%%%%%%%%%%%%%%%%%%%%%%%%%%%%%%%%%%%%%%%%%%%%%%%%%%%%%%%%%%%%%%%%%%%%%%%%
% Multicol Settings
%%%%%%%%%%%%%%%%%%%%%%%%%%%%%%%%%%%%%%%%%%%%%%%%%%%%%%%%%%%%%%%%%%%%%%%%%%%%%%%%
\setlength{\columnsep}{0.7em}
\setlength{\columnseprule}{0mm}


%%%%%%%%%%%%%%%%%%%%%%%%%%%%%%%%%%%%%%%%%%%%%%%%%%%%%%%%%%%%%%%%%%%%%%%%%%%%%%%%
% Save space in lists. Use this after the opening of the list
%%%%%%%%%%%%%%%%%%%%%%%%%%%%%%%%%%%%%%%%%%%%%%%%%%%%%%%%%%%%%%%%%%%%%%%%%%%%%%%%
\newcommand{\compresslist}{%
\setlength{\itemsep}{1pt}%
\setlength{\parskip}{0pt}%
\setlength{\parsep}{0pt}%
}


%%%%%%%%%%%%%%%%%%%%%%%%%%%%%%%%%%%%%%%%%%%%%%%%%%%%%%%%%%%%%%%%%%%%%%%%%%%%%%
%%% Begin of Document
%%%%%%%%%%%%%%%%%%%%%%%%%%%%%%%%%%%%%%%%%%%%%%%%%%%%%%%%%%%%%%%%%%%%%%%%%%%%%%

\begin{document}

%%%%%%%%%%%%%%%%%%%%%%%%%%%%%%%%%%%%%%%%%%%%%%%%%%%%%%%%%%%%%%%%%%%%%%%%%%%%%%
%%% Here starts the poster
%%%---------------------------------------------------------------------------
%%% Format it to your taste with the options
%%%%%%%%%%%%%%%%%%%%%%%%%%%%%%%%%%%%%%%%%%%%%%%%%%%%%%%%%%%%%%%%%%%%%%%%%%%%%%
\typeout{Poster Starts}
%\background{
%  \begin{tikzpicture}[remember picture,overlay]%
%    \draw (current page.north west)+(-2em,-0em) node[anchor=north west] {\hspace{-2em}\includegraphics[height=1.1\textheight]{silhouettes_background}};
%  \end{tikzpicture}%
%}

\definecolor{ritorange}{rgb}{0.9529,0.4314,0.1294}
\definecolor{ritbrown}{rgb}{0.9725,0.9686,0.9294}
\definecolor{ritbrownl}{rgb}{0.9765,0.9765,0.9451}
\definecolor{white}{cmyk}{0,0,0,0.0}
\definecolor{black}{cmyk}{0,0,0.0,1.0}

\begin{poster}{
  % Show grid to help with alignment
  grid=no,
  % Column spacing
  colspacing=1em,
  % Color style
  bgColorOne=white,
  bgColorTwo=white,
  borderColor=black,
  headerColorOne=ritorange,
  headerColorTwo=ritorange,
  headerFontColor=white,
  boxColorOne=white,
  boxColorTwo=white,
  % Format of textbox
  textborder=roundedleft,
  % Format of text header
  eyecatcher=no,
  headerborder=open,
  headerheight=0.08\textheight,
  headershape=roundedright,
  headershade=plain,
  headerfont=\Large\textsf, %Sans Serif
  boxshade=shade-lr,
  background=plain,
  linewidth=1pt
  }
  % Eye Catcher
  {} % No eye catcher for this poster. If an eye catcher is present, the title is centered between eye-catcher and logo.
  % Title
  {\sf %Sans Serif
  %\bf% Serif
  Distrivia: A Distributed Trivia Game}
  % Authors
  {\sf %Sans Serif
  % Serif
  \newline
  Brian Gianforcaro, Steven Glazer, Samuel Milton
  - Rochester Institute Of Technology
  }
  % University logo
  {{\begin{minipage}{18em}
    \hfill
    \includegraphics[scale=0.05]{logo}
  \end{minipage}}
  }

  %\tikzstyle{light shaded}=[top color=baposterBGtwo!30!white,bottom color=baposterBGone!30!white,shading=axis,shading angle=30]

  % Width of left inset image
     \newlength{\leftimgwidth}
     \setlength{\leftimgwidth}{0.78em+8.0em}

%%%%%%%%%%%%%%%%%%%%%%%%%%%%%%%%%%%%%%%%%%%%%%%%%%%%%%%%%%%%%%%%%%%%%%%%%%%%%%
%%% Now define the boxes that make up the poster
%%%---------------------------------------------------------------------------
%%% Each box has a name and can be placed absolutely or relatively.
%%% The only inconvenience is that you can only specify a relative position 
%%% towards an already declared box. So if you have a box attached to the 
%%% bottom, one to the top and a third one which should be in between, you 
%%% have to specify the top and bottom boxes before you specify the middle 
%%% box.
%%%%%%%%%%%%%%%%%%%%%%%%%%%%%%%%%%%%%%%%%%%%%%%%%%%%%%%%%%%%%%%%%%%%%%%%%%%%%%
    %
    % A coloured circle useful as a bullet with an adjustably strong filling
    \newcommand{\colouredcircle}[1]{%
      \tikz{\useasboundingbox (-0.2em,-0.32em) rectangle(0.2em,0.32em); \draw[draw=black,fill=baposterBGone!80!black!#1!white,line width=0.03em] (0,0) circle(0.18em);}}

%%%%%%%%%%%%%%%%%%%%%%%%%%%%%%%%%%%%%%%%%%%%%%%%%%%%%%%%%%%%%%%%%%%%%%%%%%%%%%
  \headerbox{Project Motivation}{name=motivation,column=0,row=0}{
%%%%%%%%%%%%%%%%%%%%%%%%%%%%%%%%%%%%%%%%%%%%%%%%%%%%%%%%%%%%%%%%%%%%%%%%%%%%%%
Our motivation for this project was to build a system that would allow people to compete across multiple platforms reliably.
High availability was a major concern for us, as it enables people to compete at all times.
This form of trivia is a multi player social game.
It takes advantage of a number of different platforms so that users can feel free to play on any device they prefer.
The web client ensures that almost any device will be able to connect and play.
Designing it on mobile platforms allows us to take advantage of the specific platform and lets people easily play on the go.
  }

%%%%%%%%%%%%%%%%%%%%%%%%%%%%%%%%%%%%%%%%%%%%%%%%%%%%%%%%%%%%%%%%%%%%%%%%%%%%%%
  \headerbox{Architecture}{name=arch,column=3}{
%%%%%%%%%%%%%%%%%%%%%%%%%%%%%%%%%%%%%%%%%%%%%%%%%%%%%%%%%%%%%%%%%%%%%%%%%%%%%%

	\includegraphics[scale=0.45]{images/amazon-logo.png}
	\includegraphics[scale=0.045]{images/android-logo.png}
	\includegraphics[scale=0.35]{images/flask-logo.png}
	\includegraphics[scale=0.5]{images/ios-logo.png}
	\includegraphics[scale=0.6]{images/logo-riak.png}
	\includegraphics[scale=0.2]{images/python-logo.png}
	\includegraphics[scale=0.5]{images/tornado-logo.png}
	\includegraphics[scale=0.3]{images/nginx-logo.png}

  }

%%%%%%%%%%%%%%%%%%%%%%%%%%%%%%%%%%%%%%%%%%%%%%%%%%%%%%%%%%%%%%%%%%%%%%%%%%%%%%
  \headerbox{Design}{name=design,column=1, span=2}{
%%%%%%%%%%%%%%%%%%%%%%%%%%%%%%%%%%%%%%%%%%%%%%%%%%%%%%%%%%%%%%%%%%%%%%%%%%%%%%
  \centering
	\includegraphics[scale=0.4]{../papers/phase-three/diagram.png}

\begin{multicols}{3}
\begin{itemize}
\item Round Robin DNS \\
\item Two Load Sharers 
\item Three front end node  \\
\item Two hot spares 
\item Riak on each server \\
\item Webapp on each server
\end{itemize}

\end{multicols}

  }

%%%%%%%%%%%%%%%%%%%%%%%%%%%%%%%%%%%%%%%%%%%%%%%%%%%%%%%%%%%%%%%%%%%%%%%%%%%%%%
  \headerbox{Implementation}{name=implementation,column=1,below=design,span=2}{
%%%%%%%%%%%%%%%%%%%%%%%%%%%%%%%%%%%%%%%%%%%%%%%%%%%%%%%%%%%%%%%%%%%%%%%%%%%%%%


\begin{multicols}{3}
\includegraphics[scale=0.22]{../papers/phase-three/iPhone_round.png} \\
\includegraphics[scale=0.52]{../papers/phase-three/android/answer2.png} \\
\includegraphics[scale=0.40]{../papers/phase-three/web_round.png}
\end{multicols}

\begin{multicols}{3}
\centering iPhone Client \\
\centering Android Client \\
\centering Web Client
\end{multicols}

  }

%%%%%%%%%%%%%%%%%%%%%%%%%%%%%%%%%%%%%%%%%%%%%%%%%%%%%%%%%%%%%%%%%%%%%%%%%%%%%%
  \headerbox{Lessons Learned}{name=lessons,column=0, below=motivation}{
%%%%%%%%%%%%%%%%%%%%%%%%%%%%%%%%%%%%%%%%%%%%%%%%%%%%%%%%%%%%%%%%%%%%%%%%%%%%%%
\begin{itemize}
  \item You will always be balancing between Durability, Performance, and Availability.
  \item Consistent deployment of system code across all servers and hot spares
  is key to being able to debug the system.
  \item Developing a consistent API before starting development is invaluable
  to reducing wasted man hours.
\end{itemize}
  }

  \headerbox{Future Work}{name=future,column=0, below=lessons}{

\begin{itemize}
\item Blackberry, iPad, Android tablet and Palm clients.
\item Categorized questions, so you can play your strengths or weaknesses.
\item Allowing users to add custom questions and answers.
\item Private local games over bluetooth or local WiFi.
\end{itemize}
%%%%%%%%%%%%%%%%%%%%%%%%%%%%%%%%%%%%%%%%%%%%%%%%%%%%%%%%%%%%%%%%%%%%%%%%%%%%%%
  }
\end{poster}%
%
\end{document}
