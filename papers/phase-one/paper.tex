\documentclass{dependencies/acm_proc_article-sp}

\usepackage{url}
\usepackage{color}
\usepackage{verbatim}
\usepackage{listings}
\lstset{
  language=C++,             % choose the language of the code
  basicstyle=\small,       % the size of the fonts that are used for the code
  numbers=left,                   % where to put the line-numbers
  numberstyle=\footnotesize,      % the size of the fonts that are used for the line-numbers
  stepnumber=0,                   % the step between two line-numbers. If it is 1 each line will be numbered
  numbersep=5pt,                  % how far the line-numbers are from the code
  backgroundcolor=\color{white},  % choose the background color. You must add \usepackage{color}
  showspaces=false,               % show spaces adding particular underscores
  showstringspaces=false,         % underline spaces within strings
  showtabs=false,                 % show tabs within strings adding particular underscores
  %frame=single,                   % adds a frame around the code
  tabsize=2,              % sets default tabsize to 2 spaces
  captionpos=t,                   % sets the caption-position to bottom
  breaklines=false,        % sets automatic line breaking
  breakatwhitespace=false,    % sets if automatic breaks should only happen at whitespace
  escapeinside={\%}{)}          % if you want to add a comment within your code
}

% Get rid of the permission block
\makeatletter
\let\@copyrightspace\relax
\makeatother

\begin{document}

%\title{ Distributed Systems: Project Description }
\title{ Distrivia: A Distributed Trivia Game }
\numberofauthors{3}
\author{
\alignauthor
Brian Gianforcaro \\
       \affaddr{Rochester Institue of Technology}\\
       \email{bjg1955@rit.edu}
\alignauthor
Steven Glazer \\
       \affaddr{Rochester Institue of Technology}\\
       \email{sfg6126@rit.edu}
\alignauthor
Samuel Milton \\
       \affaddr{Rochester Institue of Technology}\\
       \email{srm2997@rit.edu}
}
\maketitle

%\begin{abstract}
%In this paper we detail our initial idea for a distributed trivia game.
%We will explain general game play idea's as well as an example of the
%possible architecture for our system.
%\end{abstract}

\section{What We've Done}
So far, our work has mainly focused on getting everything setup to work as a
distributed system. Getting the ground work done (for example servers and load
distributers running) has provided us a way to quickly develop the clients and
server as well as provided us with a way to reliably connect the servers to the
clients.

\subsection{Servers}
Currently, we have 4 servers running. One server is being used as a web-monitor
host. We are able to login to the web-monitor to see the status of all our
servers, databases, and load distributers. This will help us manage if the
services are running and allow us to track down any problems, if they arise.
The other three servers are all running our server application and hosting the
databases. Each server contains the server application that responds to clients
as well as the Riak databases. Riak auto-synchronizes the databases between the
3 servers. If we were to add a server, we would just need to updatrver as well
as provided us with a way to reliably connect the servers to the clients.

\subsection {Traffic}
We have two load distributers set up. Both distributers are currently
configured to divide the requests among the 3 active servers. The duplicate
distributers allow for a load distributer to go down and we would still be able
to serve requests to the servers. These load distributers are selected by the
clients using round robin DNS where distrivia.lame.ws will select either load
distributer that is up at random. This provides us will stability for access to
the servers. The load distributers are set up the attempt to connect to a
server twice before deeming it dead and removing it from its list of servers to
direct clients towards.

\subsection {Clients}
Our focus for clients has been on a web-based client and an android
application. Both clients have most of the their screens completed. The
web-based client will be run from desktop platforms and laptop platforms. The
android client will run on any android phone and does not use any unique
services that require specific hardware.

\section{What's Left}
\subsection{Web Client}
Joining/playing private games
In-game leaderboard
Finish connecting to actual serversver API
Implement useful user warnings

\subsection{Android}
Connecting to server API
Registration
Private games
Leaderboards
Allowing users to control screens

\subsection{Server}
Passwords
Private game joining
Global leaderboard
Finish setting up SSL on load distributers
Set up 2 hot spares

\section {Meeting the Requirements}
-3 Devices (Android, Desktop, Laptop)
-Reliable / Flexible communication (Multiple servers, multiple load
distributers, 2 hot spares, proper communication to user)
-Coordination/Cooperation (Synchronous Databases and Rounds consisting of many
players)
-Replication / Availability / Consistency (Riak handles synchronizing databases
among servers. Load distributers provide availability)
- Naming / Service Discovery
- Fault tolerance / Handling of partial failures
- Security (SSL)
- Scalability (add new nodes and add to riak cluster and it's all set)
- Use of an interesting distributed algorithm

\newpage
%
% The following two commands are all you need in the
% initial runs of your .tex file to
% produce the bibliography for the citations in your paper.
%\bibliographystyle{abbrv}
%\bibliography{bibliography}  % sigproc.bib is the name of the Bibliography in this case
% You must have a proper ".bib" file
%  and remember to run:
% latex bibtex latex
% to resolve all references
%
% ACM needs 'a single self-contained file'!
%
%APPENDICES are optional
\balancecolumns
% That's all folks!
\end{document}
